\section{مروری بر منابع}\label{sec:sources}

\subsection{مقدمه}\label{subsec:sources_subject}
در ابتدا ویژگی‌های یک درگاه ارتباط مناسب بررسی شده و سپس برخی از درگاه‌های شناخته‌شده‌ مورد بررسی قرار می‌گیرند که چه میزان از ویژگی‌های یک درگاه ارتباط مناسب را دارا هستند و در نهایت دلایل برتری راه‌حل پیشنهادی به دیگر درگاه‌ها ذکر شده است.


\subsubsection{درگاه ارتباط مناسب}\label{subsec:sources_gateway}
مهم‌ترین ویژگی‌های یک درگاه ارتباط مناسب عبارتند از:

\begin{itemize}
    \item مسیردهی \LTRfootnote{Routing}
    \item قابلیت پیکربندی با روش‌های مختلف
    \item مقیاس‌مندی \LTRfootnote{Scalability}
    \item گسترش‌پذیری \LTRfootnote{Extendability}
    \item توازن بار \LTRfootnote{Load Balancing}
    \item ثبت رخداد‌ها \LTRfootnote{Logging events}
    \item رصد وضعیت سامانه \LTRfootnote{Monitoring}
\end{itemize}

\subsubsection{مسیر‌دهی}
اساسی‌ترین نیاز یک درگاه ارتباطی، مسیر‌دهی با استفاده از مولفه‌های مختلف درخواست از جمله نوع پروتکل، مسیر درخواست، سرتیتر‌های درخواست و... است. تنوع پشتیبانی از مولفه‌های مختلف در مسیر‌دهی باعث می‌شود که درخواست‌ها دقیق‌تر به خدمت مورد‌نظر برسند.

\subsubsection{قابلیت پیگربندی با روش‌‌های مختلف}
با توجه به محیط‌های مختلف برای بارگذاری نرم‌افزارها و تغییر پی‌در‌پی نیازها، درگاه ارتباط باید اطلاعات مورد‌نیاز برای مسیر‌دهی به درخواست‌ها را از روش‌های مختلف کشف و دریافت کند. در این‌صورت در صورت اضافه و یا حذف کردن یک خدمت، نیازی به تغییر در پیکربندی نبوده و درگاه پیکربندی خود را با تغییرات جدید به روز می‌کند.

\subsubsection{مقیاس‌مندی}
با بالا رفتن تعداد درخواست‌ها ممکن است که درگاه ارتباط تبدیل به تک نقطه‌ی شکست
\LTRfootnote{Single Point of Failure}
شود. با قابلیت مقیاس‌مندی این امکان وجود دارد که به صورت هم‌زمان چند درگاه ارتباط مختلف را اجرا کرد و هر‌کدام از این درگاه‌ها به صورت مستقل به درخواست‌ها پاسخ دهند.

\subsubsection{گسترش‌پذیری}
درگاه‌های ارتباط، معمولا به صورت پیش‌فرض طیف نیازهای عمومی تمام سامانه‌ها را پوشش می‌دهند. یکی از قابلیت‌های مناسب برای درگاه ارتباط، امکان گسترش‌پذیری و اضافه کردن نیاز مختص به یک سامانه به درگاه ارتباط است.

\subsubsection{توزان بار}
برای جلوگیری از ایجاد تک نقطه‌ی شکست در معماری میکروسرویس، چند نسخه از یک خدمت اجرا می‌شود. درگاه ارتباط باید بتواند با روش‌های مختلف، بار ورودی به سامانه را بین این نسخه‌ها تقسیم کند.

\subsubsection{ثبت رخداد‌ها و رصد وضعیت سامانه}
در معماری میکروسرویس یک سامانه از اجزای مستقل و مختلفی تشکیل شده است که هر‌کدام وظیفه‌ی مشخصی دارند. یکی از چالش‌های موجود در این معماری رصد کردن خدمت‌های مختلف است. با توجه به اینکه درگاه ارتباط با تمام خدمت‌های یک سامانه در ارتباط است می‌تواند وضعیت تمام خدمت‌ها را در قالب‌های مختلف گزارش کند.


\subsection{مروری بر ادبیات موضوع}\label{subsec:sources_literature}
در این بخش دو درگاه ارتباط محبوب و پرکاربرد Traefik و Kong مورد ارزیابی واقع می‌شوند و مزایا و معایب هر یک مشخص می‌شود.

\subsubsection{درگاه ارتباط Traefik}
درگاه ارتباط Traefik یکی از محبوب‌ترین درگاه‌های ارتباط متن‌باز است که اکثر ویژگی‌های یک درگاه ارتباط مناسب را دارا است. این درگاه ارتباط، با زبان برنامه‌نویسی Golang توسعه داده شده است که این امر باعث بهینگی و عملکرد مناسب آن شده است.

مهم‌ترین ویژگی \lr{Traefik}، سازگاری آن با محیط‌های مختلف بارگذاری نرم‌افزار است. این درگاه ارتباط، به صورت خودکار پیکربندی تمام خدمات را از محیط نرم‌افزار استخراج می‌کند که این امر باعث سادگی در استفاده از آن می‌شود.

بزرگ‌ترین مشکل Traefik عدم گسترش‌پذیری آن است. به صورتی که در صورت نیاز به اضافه کردن یک بخش اختصاصی به آن، راهی به غیر از تغییر در متن وجود ندارد.

\subsubsection{درگاه ارتباط Kong}
Kong درگاه ارتباطی است که بر روی Nginx استوار شده است و در دو نسخه‌ی متن‌باز و تجاری قابل استفاده است. این درگاه ارتباط با استفاده از Nginx به عنوان هسته‌ی خود، قابلیت اتکای بالایی دارد و از تمام مزایای آن استفاده می‌کند.

یکی از مهم‌ترین ویژگی‌های \lr{Kong}، رابط کاربری و پنل مدیریت آن است، که تغییرات در هنگام اجرا را میسر می‌سازد. پنل مدیریت Kong از ویژگی‌های نسخه‌ی
تجاری آن است.

این درگاه ارتباط،‌ با زبان برنامه‌نویسی Lua توسعه داده شده است که با توجه به محبوبیت کم‌تر این زبان، باعث شده‌ است جامعه‌ی برنامه‌نویسی کم‌تر به سراغ توسعه‌ی آن برود. از طرفی گسترش‌پذیری \lr{Kong}، بدون تغییر در متن آن، امری ممکن اما دشوار است.

\subsection{نتیجه‌گیری}\label{subsec:sources_results}
با توجه به بررسی‌های انجام شده،‌ خلا یک درگاه ارتباط با قابلیت گسترش‌پذیری بالا حس می‌شود. معماری این درگاه باید به صورتی باشد که بتوان بدون نیاز به ایجاد تغییر در متن، منطق خاص مورد استفاده در یک سامانه را به آن اضافه کرد.

هم‌چنین با توجه به تنوع در محیط‌های بارگذاری سامانه‌های با معماری میکروسرویس، درگاه ارتباط باید بتواند با اجزاهای محیطی مختلف ارتباط برقرار کرده و تنظیمات مربوط به خدمت‌های مختلف را از این محیط‌ها دریافت کند. نمونه‌هایی از محیط‌های مختلف عبارتند از سکوی
\LTRfootnote{Platform}
\lr{Docker}، \lr{Kubernetes} و... .

درگاه ارتباط پیاده‌سازی شده با تمرکز بر گسترش‌پذیری و قابلیت پیکربندی به عنوان شاخص‌های اصلی نسبت به سایر درگاه‌های موجود پیاده‌سازی شده است. جزئیات مربوط به نحوه‌ی پیاده‌سازی این ویژگی‌ها در فصل
\ref{sec:implementation}
آمده است.

مقایسه‌ی درگاه‌های ارتباط مختلف، در جدول
\ref{tab:gateways}
قابل مشاهده است.


\begin{table}[H]
    \centering
    \caption{مقایسه‌ی درگاه‌های ارتباط مختلف}\label{tab:gateways}
    \begin{tabular}{|c|c|c|c|}
        \hline
         & Traefik & Kong & درگاه ارتباط پیشنهادی\\
        \hline
        مسیردهی & بالا & بالا & بالا\\
        \hline
        قابلیت پیکر‌بندی با روش‌های مختلف & بالا & متوسط & بالا\\
        \hline
        مقیاس‌مندی & بالا & بالا & بالا\\
        \hline
        گسترش‌پذیری & پایین & متوسط & بالا\\
        \hline
        توازن بار & بالا & بالا & متوسط\\
        \hline
        ثبت رخداد‌ها & بالا & بالا & متوسط\\
        \hline
        رصد وضعیت سامانه & بالا & متوسط & بالا\\
        \hline
        سادگی استفاده & بالا & متوسط & بالا\\
        \hline
    \end{tabular}
\end{table}

\cleardoublepage 