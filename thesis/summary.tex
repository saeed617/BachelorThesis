\section*{چکیده}
\addcontentsline{toc}{section}{\numberline{} چکیده‌}
با توجه به رشد روز‌افزون کسب و کارهای مجازی و نیاز به ثبات و کارایی بالا در تامین درخواست‌‌های مشتریان، معماری‌های جدیدی برای توسعه نرم‌افزار مورد استفاده قرار می‌گیرند. ویژگی اصلی این معماری‌ها، مقیاس‌مندی و تاب‌آوری بالا برای پاسخ به مشتریان است.

معماری میکروسرویس، یکی از محبوب‌ترین معماری‌های جدید است. تمرکز این معماری بر مقیاس‌مندی، گسترش‌پذیری و عدم وابستگی خدمت‌های مختلف یک سامانه به یکدیگر است. نحوه‌ی دستیابی این معماری به ویژگی‌های ذکر شده، جداسازی عملکرد‌های مختلف سامانه به چند قسمت و اجرای خودمختار هر یک از این قسمت‌ها در یک محیط جدا‌شده
\LTRfootnote{Isolated}
است.

همانند بسیاری از معماری‌های دیگر، پیاده‌سازی درست معماری میکروسرویس، دارای چالش‌های مختلفی است. با توجه به ماهیت غیر متمرکز سامانه‌ها در این معماری، اتصال کاربران به خدمت‌های ارائه‌شده به سادگی گذشته نخواهد بود. زیرا کاربر برای استفاده از خدمت‌های مختلف نیاز به دریافت اطلاعات از قسمت‌های مختلف سامانه را دارد. از طرفی تغییر روش ارتباط کاربران با سامانه و به‌روزرسانی نرم‌افزارهای سمت کاربر، بسیار پرهزینه خواهد بود.

الگوی درگاه ارتباط با رابط‌های برنامه‌نویسی جهت حل این چالش طراحی شده است. در این الگو، کاربران همانند سابق درخواست‌های خود را تنها به یک سامانه‌ی میانی ارسال می‌کنند. وظیفه‌ی این سامانه‌ی میانی، دریافت درخواست و هدایت ‌آن به سمت خدمت‌های مرتبط به درخواست است.

علی‌رغم وجود درگاه‌های ارتباط متن‌باز و تجاری مختلف، اکثر آن‌ها گسترش‌پذیری و قابلیت پیکربندی ضعیفی دارند. از این‌رو پیاده‌سازی یک درگاه ارتباط که خلاهای موجود را پوشش دهد، تمرکز اصلی این پروژه است.

\noindent {\bf کلید واژه‌ها:}
درگاه ارتباط با رابط‌های برنامه‌نویسی، معماری میکروسرویس، سامانه‌های غیر متمرکز، سامانه‌ی گسترش‌پذیر
\cleardoublepage 